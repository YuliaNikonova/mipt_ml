\documentclass{article}
\usepackage [utf8] {inputenc}
\usepackage [T2A] {fontenc}
\usepackage{ amssymb }
\usepackage{ alltt }
\usepackage{ amsmath }
\usepackage{ tikz }
\usepackage{neuralnetwork}
\usepackage[linesnumbered,boxed]{ algorithm2e }
\usetikzlibrary{matrix,chains,positioning,decorations.pathreplacing,arrows}
\title{Машинное обучение \\
Теоретическое задание №1}
\author{Никонова Юлия}
\date{}
\begin{document}
\maketitle
\textbf{Задача 1.} Стоит отдать предпочтение первому алгоритму(наивный байесовский классификатор), так как во втором случае, скорее всего, имело место переобучение, то есть сотрудник получил алгоритм, лучше всего подходящий именно для этой тестовой выборки, а не дающий более хорошее качество в среднем.
\newline
\textbf{Задача 2.}
Формула для оценки $leave-one-out$:
\begin{equation*}
loo = \frac{1}{l} \sum_{j \in B_{r(x_i)} \setminus i}
y_j \, - \, y_i)^2 = \frac{1}{l} \sum_{i=1}^{l}(\frac{1}{n_{r(x_i)}-1})^2(\sum_{j \in B_{r(x_i)} \setminus i} y_j \, - \, y_i)^2
\end{equation*}
\begin{equation*}
(\sum_{j \in B_{r(x_i)} \setminus i} y_j \, - \, y_i)^2 = (\sum_{j \in B_{r(x_i)} \setminus i} y_j)^2 \, - \, 2 \sum_{j \in B_{r(x_i)} \setminus i} y_j \cdot y_i \, + \, y_i^2
= ( \sum_{j \in B_{r(x_i)}} y_j - y_i)^2 - 2 (\sum_{j \in B_{r(x_i)} } y_j -y_i) \cdot y_i \, + \, y_i^2
\end{equation*}
Таким образом, 
\begin{equation*}
loo = \frac{1}{l} \sum_{i=1}^{l}\frac{1}{n_{r(x_i)}-1} [( \sum_{j \in B_{r(x_i)}} y_j - y_i)^2 - 2 (\sum_{j \in B_{r(x_i)} } y_j -y_i) \cdot y_i] + \sum_{i=1}^{l}(\frac{1}{n_{r(x_i)}-1} y_i^2)
\end{equation*}
Докажем, что вычисление этой формулы требует не более $O(l)$ операций.
Очевидно, что $\sum_{i=1}^{l}(\frac{1}{n_{r(x_i)}-1} y_i^2)$ считается за $O(l)$.
Введем переменные 
\begin{equation*}
Y_m= \sum_{j \in B_m } y_j, \, m \in \{ r(x_i), \, 1 \leqslant i \leqslant l \}
\end{equation*}
Очевидно, что отношение принадлежности к $B_m$ разбивает множество $[1, \cdots, l ]$ на классы эквивалентности.
\newline
Рассмотрим следующий алгоритм:
\newline
\begin{algorithm}[H]
\SetAlgoNoLine
\For{$i \leftarrow 1$ \KwTo $l$}{
\eIf{$defined(Y_{r(x_i)})$}{
continue\;
}{
$Y_{r(x_i)} \leftarrow 0$ \;
\ForEach{element $j \in B_{r(x_i)} $}
{$Y_{r(x_i)} \leftarrow Y_{r(x_i)} + y_j$}
}
}
\caption{Preprocessing}
\end{algorithm}
Каждый элемент мы прибавляем к какой-либо $Y_m$ не более одного раза, значит сложность подобного алгоритма ~--- $O(l)$.
\newline
Теперь мы можем записать оценку так:
\begin{equation*}
loo =  \frac{1}{l} \sum_{i=1}^{l}\frac{1}{n_{r(x_i)}-1} [( Y_{r(x_i)} - y_i)^2 - 2 (Y_{r(x_i)} -y_i) \cdot y_i] + \sum_{i=1}^{l}(\frac{1}{n_{r(x_i)}-1} y_i^2)
\end{equation*}
Так как все $Y_{r(x_i)}$ заранее посчитаны за $O(l)$, вычисление этой формулы потребует $O(l)$ операций.
Значит, мы можем посчитать оценку $leave-one-out$ за $O(l)$.
\newline
\textbf{Задача 3.}
\newline
\tikzstyle{sigma} = [rectangle, draw, font=\Huge, text centered, inner sep=2pt]
\begin{tikzpicture}[node distance = 2cm, auto]
\begin{scope}
\node at (0,3cm) 
  (x1) {$x_1$};
\node at (2, 3cm)
  (w1_1) {$1$};
\node at (2, 2cm)
  (w1_2) {$-1$};
\node at (0,1cm)
  (x2) {$x_2$};
\node at (2,1cm)
  (w2) {$1$};
\node at (0,-1cm) 
  (x3) {$x_3$};
\node at (2, 0cm)
  (w3_1) {$1$};
\node at (2,-1cm)
  (w3_2) {$1$};
\node at (0,-3cm)
  (x4) {$1$};
\node at (2,-2cm)
  (w4) {$\frac{1}{2}$};
\node[sigma] at (4,1cm)
  (sigma1) {$\Sigma$};
\node at (6,1cm)
  (s1) {$1$};
\node[sigma] at (4,-1cm)
  (sigma2) {$\Sigma$};
\node at (6,-1cm)
  (s2) {$1$};
\node at (6,-3cm)
  (w4_3) {$-1$};
\node[sigma] at (8, 0cm)
  (sigma3) {$\Sigma$};
\node at (12, 0cm)
  (formula) {$(\bar x_1 \vee x_3) \& (x_1 \vee x_2 \vee x_3)$};
\end{scope}
\draw[-latex] (x1) -- (w1_2);
\draw[-latex] (x1) -- (w1_1);
\draw[-latex] (x2) -- (w2);
\draw[-latex] (x3) -- (w3_2);
\draw[-latex] (x3) -- (w3_1);
\draw[-latex] (x4) -- (w4);
\draw[-latex] (x4) -- (w4_3);
\draw[-latex] (w1_1) -- (sigma1);
\draw[-latex] (w1_2) -- (sigma2);
\draw[-latex] (w2) -- (sigma1);
\draw[-latex] (w3_1) -- (sigma1);
\draw[-latex] (w3_2) -- (sigma2);
\draw[-latex] (w4) -- (sigma2);
\draw[-latex] (sigma1) -- (s1);
\draw[-latex] (sigma2) -- (s2);
\draw[-latex] (s1) -- (sigma3);
\draw[-latex] (s2) -- (sigma3);
\draw[-latex] (w4_3) -- (sigma3);
\draw[-latex] (sigma3) -- (formula);
\end{tikzpicture}

\end{document}