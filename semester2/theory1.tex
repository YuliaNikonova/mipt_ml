\documentclass{article}
\usepackage [utf8] {inputenc}
\usepackage [T2A] {fontenc}
\usepackage{ amssymb }
\usepackage{ alltt }
\usepackage{ amsmath }
\usepackage{ tikz }
\usepackage{neuralnetwork}
\usetikzlibrary{matrix,chains,positioning,decorations.pathreplacing,arrows}
\title{Машинное обучение \\
Теоретическое задание №1}
\author{Никонова Юлия}
\date{}
\begin{document}
\maketitle
\textbf{Задача 1.} Стоит отдать предпочтение первому алгоритму(наивный байесовский классификатор), так как во втором случае, скорее всего, имело место переобучение, то есть сотрудник получил алгоритм, лучше всего подходящий именно для этой тестовой выборки, а не дающий более хорошее качество в среднем.

\textbf{Задача 2.}

\tikzstyle{block} = [circle, draw, fill=blue!20, text centered, inner sep=2pt]
\begin{tikzpicture}[node distance = 2cm, auto]
\begin{scope}[start chain=1]
\node[on chain=1] at (0,1.5cm) 
  (x1) {$x_1$};
\node[on chain=1,join=by o-latex] 
  (w1_1) {$-1$};
\end{scope}
\begin{scope}[start chain=2]
\node[on chain=2] at (0,0) 
  (x2) {$x_2$};
\node[on chain=2,join=by o-latex] 
  (w2_1) {$-1$};
\node[block, on chain=2,join=by o-latex] 
  (sigma1) {$\Sigma$};
\end{scope}
\begin{scope}[start chain=3]
\node[on chain=3] at (0,-1.5cm) 
  (x3) {$x_3$};
\node[on chain=3,join=by o-latex] 
  (w3_2) {$-1$};
\node[block, on chain=3,join=by o-latex] 
  (sigma2) {$\Sigma$};
\end{scope}
\begin{scope}[start chain=4]
\node[on chain=4] at (0,3cm) 
  (const) {$1$}
\node[on chain=4,join=by o-latex] 
  (w4_1) {$-1$};
\end{scope}
\end{tikzpicture}
\end{document}